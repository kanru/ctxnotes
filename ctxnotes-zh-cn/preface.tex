\startcomponent ch1
\product ctxnotes

\chapter{对 \TEX\ 的一些认知}
\bookmark{对 TeX 的一些认知}

本章可读,可不读。读,是为了让自己更明白 \TEX\ 是什么,并且了解一下 \CONTEXT\ 的概况。不读,是因为取这样标题的章节通常包含了许多废话。

“\TEX\ 是什么”这一节,是 Hans Hangen 写的\footnote{原文见:\type{http://www.tug.org/pracjourn/2005-3/walden-whatis/hagen-4.pdf}},我只是按照自己的理解翻译了过来。之所以让它在本章出现,是因为我认为作为 \CONTEXT\ 的主要开发者,Hans Hagen 对 \TEX\ 具有深入客观的认识,他有资格告诉我们一些 \TEX\ 的真实面目,而不是那些在网络上流传的诸多神话。另外,在本章中,我还记述了自己对 \CONTEXT\ 的一些认识,并对这份文档的写作意图进行些许交代。

\section{\TEX\ 是什么}
\bookmark{TeX 是什么}

在此,我认真思考了一些人提出的一些问题。这篇文章与其说它是一份评论,倒不如说它像一个开关,可以触发更为深刻地讨论。如果你只是想知道什么是 \TeX,你可以直接读“\TeX\ 是什么……以及我为什么喜欢它”那一节。

\subsubject{所有的 \TeX\ 都是平等的……\\ \rightaligned{……只是有些 \TeX\ 比其他 \TeX\ 更加平等而已}}

对于什么是 \TeX\ 这一问题实在很难给出明确的答案。例如 Peter Flom 使用“\LaTeX\ 是……”作为开场白,将 \TeX\ 程序等价为 \LaTeX\ 这个宏包。这种等价随处可见,因此许多用户并不知晓 pdf\TeX(程序) 与 pdf\LaTeX(宏包)的区别。在许多系统上,如果没有显示的指明 pdf\TeX\ 所使用的宏包,那么它会调用 plain \TeX\ 格式,这听起来就更让人混乱了。

\TeX\ 的双重身份也常常令人混乱,它即是一种排版语言,又是一个解释器/排版工具。\TeX Book 这本书中不仅讲述了 \TeX\ 的双重身份,还涉及了 plain \TeX\ 格式。因此,对于 \TeX\ 而言,我们有必要区分:

\definition{语言} \TeX\ 排版命令,由基本命令与宏命令构成,还有一些扩展命令,如 e\TeX\ 添加的命令\par
\definition{程序} \TeX(最早的\TeX\ 解释器)、pdf\TeX、\XeTeX、Lua\TeX\ 等 \TEX\ 语言解释器\par
\definition{宏包} Plain \TeX、\AMSTEX、\LaTeX、\LAMSTEX、\ConTeXt\ 等基于语言实现的宏命令集

\LaTeX\ 用户有多种调用 \TeX\ 程序的方式:

\definition{latex} 可预装入 \LaTeX\ 宏包的 (pdf)\TeX\ 引擎\par
\definition{pdflatex} 同上,不过它是直接输出 pdf 文档\par
\definition{xelatex} 由 \XeTeX\ 引擎预装入的 \LaTeX\ 宏包\par
\definition{lambda} 由 ALEPH 或 OMEGA 引擎预装入的 \LaTeX\ 宏包

对于 \ConTeXt\ 用户而言,\LaTeX\ 调用 \TeX\ 程序的经验就用不上了,因为 \ConTeXt\ 提供了 texexec 脚本,可以使用统一的方式实现不同 \TeX\ 程序的调用,譬如:

\definition{texexec --pdf somefile.tex} 使用预装入 \ConTeXt\ 宏包的 pdf\TeX\ 引擎\par
\definition{texexec --xtx somefile.tex} 同上,不过这次调用的是 \XeTeX

很久以来,\TeX\ 引擎产生 DVI 格式输出,然后借助一些程序可以将其处理成 PostScript 文档(适合打印或屏幕阅读)。由于 PDF 文档格式日趋流行,因此有必要再使用一些工具将 PostScript 文档转化为 PDF 文档。现在,使用 pdf\TeX\ 引擎可以直接由 \TeX\ 源文档生成 PDF 文档,所以也就没有必要再延用过去的那种繁琐的文档生成方式了。

无论你怎样使用 \TeX,都应当搞清楚你所调用的命令的相关术语的确切概念,否则当你你与他人进行 \TeX\ 知识交流或者去 BBS 寻求帮助时,别人未必使用与你相同的命令。比如在较早的 \TeX\ 发行套件中,pdf\TeX\ 默认输出的文档格式是 DVI,除非你显示指定 PDF 格式输出;而在新的 \TeX\ 发行套件中,pdf\TeX\ 默认输出的文档格式是 PDF。怎么样,现在还感到困惑么?

\subsubject{\TeX\ 可以生成精美的文档……\\\rightaligned{……但它不会给你什么承诺}}

有一些 \TeX\ 的拥护者,他们为了让更多的人接受\TeX,便经常地赞美 \TeX\ 排版的精美。这种行为过于一厢情愿了。TeX排版的精美是没有什么疑问的,但是许多文档看上去很 \TeX\ 化,就像 MS Word 文档看上去很 word 化,QUARK 文档看上去很 quark 化。在风格的变化方面,字体及其格式的影响不是很大,而预定义的模板被重复地使用,这些都导致了文档的版面千篇一律。例如,有些 \TeX\ 用户经常取笑使用 PowerPoint 演示文档(因为通过这些演示文档的一些特征就可以辨识出来这是 PPT 文档),却从未意识到他们自己也是那么干的。许多 \TeX\ 用户宣称 \TeX\ 可以有效地自动处理段落文本断行问题,但是他们往往又不得不对那些很滑稽的行间距做出宽容的姿态,这些行间距通常是自己所使用的一些命令与那些尝试一揽子解决所有问题的宏包发生冲突时所致。由于 \TeX\ 在文本对齐方面做的非常出色,因此一旦文档的版面发生单词越出边界的现象,就会非常得显眼,在网络上张贴的许多文档经常向我们展示这一“特效”。

\subsubject{\TeX\ 是易于使用的……\\\rightaligned{……但需要你付出努力}}

\TeX\ 能够很聪明地处理图形与字体,但是邮件列表里 (BBS) 的诸多帖子昭示着这并非是小事情,即便对于那些 \TeX\ 老手而言。\TeX\ 可以是一个易于使用的排版系统,但是用户如果要真正的驾驭它,也必须要经历一个痛苦的过程。有些东西纯粹是排版技巧,无关乎你使用的是哪一种排版系统。

使用 \TeX\ 能得到一个极大的好处,那就是 \TeX\ 用户能够普遍地热心帮助新手。由于 \TeX\ 用户大都出于自己的意愿选择了 \TeX,因而他们通常也能够付出足够多的努力来掌握它。

有关 \TeX\ 及其宏包的用法,我们能够在网络或书店里找到许多的学习资源(手册、答疑、Wiki、邮件列表、新闻组等等)。

\subsubject{成也 \TeX……\\\rightaligned{……败也 \TeX}}

在许多计算机语言中,程序员不得不明确地告诉机器有一些文本要输出。但是 \TeX\ 却不同,\TeX\ 对于文本的任何修饰都可能会变成可见的。有人嘲笑 MS Word 排版的文档会出现一些不协调的空区,譬如一些重复的空格。事实上,使用 \TeX,如果你不了解一些宏标记的具体细节,也很容易引入许多很可笑的空区,导致页面乱糟糟的。总之,在做一些宣判或开一些玩笑时,一定要小心谨慎才是。

在讨论 MS Word 与 \TeX\ 的区别时,有人说 MS Word 从来也搞不清楚在何种情况下需要切换字体,比如一个粗体显示的单词之前的空格是否也要粗体显示?但实际上 \TeX\ 在这方面并不比 MS Word 高明多少。

我们来看一下在 \TeX\ 中如何能让一个段落变的狭窄一些,如下:

\startTEX
\def\StartNarrow{\bgroup\leftskip1em\rightskip1em\relax}
\def\StopNarrow {\egroup}
\StartNarrow
some lines of text
\StopNarrow
\stopTEX

像上面这样的处理貌似可以将段落文本放置于一个已经作了限制的区域中。但实际上,上面的代码并不能得到一个狭窄的段落,除非你显式地添加上段落终止标记:

\startTEX
\StartNarrow
some lines of text\par
\StopNarrow
\stopTEX

对于这一问题,最好的解决方案是修改 \type{\StopNarrow} 宏定义:

\startTEX
\def\StartNarrow{\bgroup\leftskip1em\rightskip1em\relax}
\def\StopNarrow {\par\egroup}
\stopTEX

有许多间距都与采用这种方式得到的特征以及并不总是很清楚的代码效果有关。适用于这一份文档的样式并不见得就适合其它文档。所有的一切都依赖于你的 \TeX\ 是如何设定的以及宏包的作者如何协调好他们的工作。

\subsubject{\TeX\ 是稳定不变的……\\\rightaligned{……汗……我们真的希望是这样?}}

Don Knuth 一厢情愿地认为 \TeX\ 程序的功能能够适应性地扩展,去解决那些当前解决不了的问题。在优秀的老 \TeX\ 中有两个扩展的例子:特效 (Special) 与著述 (Write)。所谓特效,就是提供一种方法去控制 \TeX\ 引擎以实现一些特殊效果,比如颜色或向图形插入之类。如果没有特效这一扩展,那么 \TeX\ 用户就要面临很大的麻烦,需要手动去做图片复本的剪切、粘帖之类的事情。\TeX\ 的著述扩展在撰写文献方面非常有用,它提供了目录、交叉引用以及其他特征,这些特征的实现都需要一种反馈回路的运行机制。这两种扩展都是以 \TeX\ 宏包的方式实现的,因此 Don Knuth 满怀信心地认为TeX引擎不需要修改,它可以通过宏包的形式不断地充实自身的功能。

事实上,有一些非 Knuth 式的扩展,但并不是很多。因为,鲜有人能不辞劳苦去写一个用于化学领域文档的子排版系统去与数学排版子系统并驾齐驱。许多扩展大都是采用 \TeX\ 宏的形式来实现的。迄今为止,没有人对文本行号统计、并行输出以及人性化及多语言兼容等方面提出健壮可靠的扩展方案。这又一次不得不借助 \TeX\ 宏开发的方式来实现,这是一种很脏的方式。你可能大呼庆幸,因为那些出版商没有这些需求。

无论 \TeX\ 引擎的实现有多么完美,总是有人希望它能够继续改进。目前,最值得称道的一些 \TeX\ 扩展程序 \ETEX、\PDFTEX、\XETEX。\ETEX\ 提供了一些额外的编程功能。\PDFTEX\ 将 \TeX\ 推到了 21 世纪,提供了边注字距调整与视觉缩放优化功能,另外还实现了完善的 PDF 输出功能。\XETEX\ 使 \TeX\ 具备了处理 Unicode 编码与 OpenType 字体的功能。实践证明,只有不断改进 \TeX\ 引擎,才可以保证 \TeX\ 不会被时代遗弃。\footnote{由于 Hans 撰写这篇文章比较早,现在又有了一个新的 \TeX\ 引擎---\LUATEX,本文便是使用基于 \LUATEX\ 引擎的 \ConTeXt\ 编译而得到的。\LUATEX\ 项目与 \XETEX\ 项目所要解决的基本问题是一样的,但是前者实现了 Lua 语言在 \TeX\ 引擎的嵌入,使得用户更灵活地扩展 \TeX\ 引擎的功能。}

当然,\TeX\ 宏包也扮演了非常重要的角色。因为无论 \TeX\ 引擎怎么变化,但这些宏包基本上还可以照常运行。这也就是说,很久以前写的 \TeX\ 文档,利用扩展之后的 \TeX\ 引擎还可以正常编译输出成适合印刷或适合屏幕阅读的文档。

有人赞美 \TeX\ 系统鲜有 bug,但是对于 23 年之前使用 \type{\leaders} 排版命令的 \TeX\ 文档,使用现在的 \TeX\ 引擎就无法再编译了。因为在这期间,\TeX\ 引擎的一些 bug 得到了修正,因此对文档的处理机制多少都有些变化,尽管这些变化非常之小。不过,通常而言,说 \TeX\ 系统鲜有 bug 也不为过,这种说法只是在你不刻意让今天的 \TeX\ 系统来处理很久很久之前的文档或使用很久以前的宏包的前提下才成立的。顺便说一下,也有一些有关程序稳定不变的例子,譬如计算机语言编译器与解释器,实际上 \TeX\ 本身就是一种计算机语言+编译器/解释器。

\subsubject{\TeX\ 是什么……\\\rightaligned{……以及我为什么喜欢它?}}

\TeX\ 是一个允许你创建属于你自己的排版环境的系统。在它所存在的这 20 多年来,出现了许多的排版环境,譬如 \LaTeX\ 与 \ConTeXt,它们让用户可以更方便的使用 \TeX。你可以根据自己的需要对它们进行功能上的扩展或者决定坚持使用它们所提供的功能,这完全取决于你个人。也有一些功能是你难以驾驭的,但这也是一个功能丰富的系统所带给你的一个必然结果。

如果你能够坚守着你所使用的排版环境的规范,比如保持你的文档源码清晰,那么你的文档就能够耐得住时间的考验。如果你采用结构化方式编辑文档,以抽象的方式定义文档的排版布局,那么最终可以实现在任何平台上都能够得到最终的编译输出结果。你可以将文档撰写任务分配给其他人,你们一起协同工作,这种协同工作方式有助于实现邮件列表的支持,形成一个气氛友好的社区、用户群体,实现书籍与手册的撰写。如果你想完全驾驭 \TeX\ 系统(譬如 \LaTeX\ 或 \ConTeXt),这需要耗费一段时间来掌握它们。学习周期过于漫长,这似乎是个难题,但是对于很对用户而言,他们终生都在使用 \TeX,并得益于此,所以学习周期的问题不再是问题。Don Knuth给予了我们创造精美文档的能力,但是你需要付出一定的努力才能够掌握它。Don Knuth 也给出了一个重要的时间界限条件,即在 100 年后,印刷技术有了极大的进步,这些使用 \TeX\ 标记所写的文档依然可以使用 \TeX\ 引擎进行有效处理。我们所写的 \TeX\ 文档,能够生存这么长久,这本身就是一件很让人舒服的事情。只是要小心,在这么漫长的时间里,你可能会碰到一些麻烦,要避免它们,就需要保持一种开放的心态去面对 \TeX\ 的缺陷、神话以及一些奇怪的解决方案。


\section{我对 \CONTEXT\ 的认知}
\bookmark{我对 ConTeXt 的认知}

\CONTEXT\ 是荷兰 Pragma-ADE 公司基于 \TEX\ 实现的一种高端文档制造工具,使用它可以制作非常精美的 PDF 文档,适用于科技文档排版与屏幕演示文档制作。与 \LATEX\ 相比,\CONTEXT\ 的开发更为集中、活跃与激进。

\CONTEXT\ 的版本可以分为 MkI、MkII 和 MkIV。MkI 的用户界面是荷兰语,并且只有开发者可以看到用户界面的具体实现。MkII 将用户界面替换成英文,并且开放了一些用户界面的实现,便于用户参与开发。MkIV 是新一代 \CONTEXT,其中许多模块重新实现了,最具革命性的是引入了 \LUATEX\ 引擎。\LUATEX\ 是 pdf\TEX\ 的一个扩展版本,其中植入了 Lua 语言,这意味着在 \TEX\ 文档中可以使用 Lua 完成一些程序,使 \TEX\ 文档演进成一种真正的文档排版编程语言。另外,\LUATEX\ 提供对本地 TTF \& OTF 字体的直接支持,对于中文用户而言,困扰大家多年的中文字体嵌入的问题算是得到很好地解决。所以,\LUATEX\ 似乎是可以结束目前 \TEX\ 引擎版本混乱、功能落后的最理想的解决方案。

事实上 \CONTEXT\ 还有一个 MkIII 版本,这是为 \XETEX\ 引擎预留的。\XETEX\ 原本是 Mac OS 平台上的一个 \TEX\ 引擎项目,不过现在 Linux、 Windows 平台都有其移植版本。\XETEX\ 所要解决的问题与 \LUATEX\ 差不多,但前者没有像后者那样提供一种内嵌的脚本语言。目前的 \XETEX\ 已经可以较为稳定地运行了,而 \LUATEX\ 还处于 Beta 版本,据说今年夏天会正式发布\footnote{在写这份文档的时候,今年夏天快要过去了。}。现在,\XETEX\ 的最新版是 0.999,已经可以支持 UTF-8 编码以及本地 TTF \& OTF 字体调用,用于中文文档处理基本上没有什么问题了。特别对于 \LATEX\ 的中文用户,由于 C\TEX\ 论坛上的 mytex、yindian 所提供的 XeCJK 与 zhspacing 宏包,已经可以让使用 \XETEX\ 引擎的 \LATEX\ 得以完美地支持中文排版。

现在,\PDFTEX\ 项目已并入 \LUATEX\ 项目中,这宣告着在今后一段很长的时间里 Knuth \TEX、\LUATEX\ 与 \XETEX\ 三足鼎立的时代的来临,不过 Knuth \TEX\ 引擎存在的意义也许仅在于兼容以前的文档或留给后人去考古或者兼容历史遗留文档。

几乎所有的 \TEX\ 发行版中都包含了 \CONTEXT\ 模块,但是若想更容易地使用 MkIV 版本,建议安装 \CONTEXT\ Wiki\footnote{\type{http://wiki.contextgarden.net}} 上提供的 Minimals (\CONTEXT\ 最小发行版)。\CONTEXT\ Minimals 仅提供了运行 \CONTEXT\ 环境所需要的软件包,所以如果你想使用 \LATEX,还请安装其它 \TEX 发行版,譬如 \TEX\ Live,只需不安装其中的 \CONTEXT\ 模块即可,因为 \CONTEXT\ Minimals 与其它 \TEX\ 发行版可以友好地共存。

\section{关于这份文档}

这份文档仅仅是我个人学习 \CONTEXT\ 过程中所获取的一些我认为比较重要的知识的汇总,它也许有些凌乱,不是面面俱到,有些知识讲述地过于浅薄甚至出现了认识错误,这都是因为我还在学习中。如果您恰好是一个有经验的 \CONTEXT\ 用户,恰好看见了错误,恰好又不吝赐教,你就是我的一字之师啊,(突然提高声音)我记你一辈子!\footnote{出自武林外传第五十回。}

由于自知能力有限,我没有将这份文档写成一部 \CONTEXT\ 中文教程的欲望,因此这份文档并不能教会你使用 \CONTEXT。官方的文档非常全面,如果想掌握 \CONTEXT,我推荐你去阅读它们。其实在这份文档中,每当我碰到一些细节知识没有耐心去讲述时,便会偷懒,往往会装作很耐心地告诉你在 \CONTEXT\ 手册的哪一节可以找到详细的记述。

我写这份文档的出发点与你之所以读到它是一样的,都是出于对 \CONTEXT\ 的喜爱,并且希望它可以化作自己的如抟之笔,用以书写优美的文档。当我开始写这份文档的时候,还是一个菜鸟。待得这份文档越来越完整,越来越丰满的时候,也许我就变成了一个有经验的菜鸟。我这么说,不是谦虚,而是面对着 \TEX\ 的博大,看见了自己的渺小。

本文档有许多内容是只适于 Linux 环境,特别是关于 \CONTEXT\ 编译环境搭建方面的内容。Windows 或 Mac OS 环境中的 \CONTEXT\ 我从未尝试过,并且也不想去尝试。在我看来,\TEX\ 在 Linux 终端里运行,再配合一些文本处理工具,俨然如鱼得水一般,所以我一点都不向往 GUI 的陆地。

\stopcomponent
