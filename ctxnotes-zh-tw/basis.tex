\startcomponent basis
\product ctxnotes

\chapter{讓 \CONTEXT\ 工作起來}
\bookmark{讓 ConTeXt 工作起來}

目前,只有 \ConTeXt\ Minimals(\ConTeXt\ 最小包)提供了最新的 ConTeXt 版本並且可以使之與系統所安裝的其它 TeX 發行版友好共存。本章介紹 \ConTeXt\ Minimals 的安裝、基本使用以及 MkIV 的中文支持等知識,目的在於讓你多快好省地擁有一個 \CONTEXT\ 文檔編譯環境。本章的最後介紹了一些初學者應當閱讀的文檔以及一些學習資源。

\section[installmkiv]{\ConTeXt\ Minimals}
\bookmark{ConTeXt Minimals}

首先,在本地機器上創建一個 \ConTeXt\ 安裝目錄,該目錄的位置與名字視個人嗜好而定,我喜歡用 \type{/opt/context}。為了不引起混亂,下文提及 \ConTeXt\ 安裝目錄皆以如下設定的 Bash 變量 \type{$CTXDIR} 代替:

\starttyping
$ export CTXDIR=/opt/context
$ mkdir -p $CTXDIR
\stoptyping

然後下載 ConTeXt 最小包安裝腳本,將其存放於 \type{$CTXDIR} 目錄並執行:

\starttyping
$ cd $CTXDIR
$ rsync -ptv \
rsync://contextgarden.net/minimals/setup/linux/first-setup.sh .
$ ./first-setup.sh
\stoptyping

由於安裝腳本需要聯網下載 \ConTeXt\ Minimals,需要等待一段時間才可以完成安裝。這段時間,你可以做點自己喜歡做的事情。

\ConTeXt\ Minimals 安裝完畢後,就開始進行 \ConTeXt\ 運行環境的配置,這需要設置許多的系統環境變量。不過莫要擔心,\ConTeXt\ Minimals 提供了一個配置腳本,使用 source(亦稱點命令)命令加載該文件即可完成所有 \ConTeXt\ 所需環境變量的設定。

\starttyping
$ . $CTXDIR/tex/setuptex $CTXDIR/tex
\stoptyping

上述命令行可將 setuptex 配置文件中的諸多環境變量的設定在當前終端環境中保持有效性。這意味著,每次使用 \ConTeXt\ Minimals,可以開一個終端窗口,然後執行一次上述命令,之後就可以在該終端中使用 \ConTeXt\ 環境。如果需要使用系統所安裝的其它 \TeX\ 發行套件,可另建一個未開啟 \ConTeXt\ 環境的終端即可。

雖然每次啟用 \ConTeXt\ Minimals 所輸入的命令行並不複雜,但是盡力讓命令行簡約且易於輸入一向是 Unix-like 系統的光榮傳統,因此我在 \type{$HOME/myscript} 目錄下建立了一個配置文件 ctx,該文件內容如下:

\starttyping
export TEXDIR=/opt/context/tex
source $TEXDIR/setuptex $TEXDIR
export OSFONTDIR="/usr/share/fonts/{adobe,winfonts}"
\stoptyping

然後將 \type{$HOME/myscript} 目錄添加到系統環境變量 \type{$PATH} 並使之生效:

\starttyping
$ echo "PATH=$PATH:$HOME/myscript" >> ~/.bashrc
$ . ~/.bashrc
\stoptyping

這樣,以後每次啟用 \ConTeXt\ 環境時,只需執行以下命令:

\starttyping
$ . ctx
\stoptyping

注意,在 ctx 文件中設置了 \type{$OSFONTDIR} 變量,該變量的意義在下一章講述 MkIV 中文排版時會進行詳細闡述,現在可不予理會。

使用 \type{ctxtools} 命令並配合 \type{-updatecontext} 參數可對所安裝的 \ConTeXt\ Minimals 進行在線升級,不過前提是要建立 \type{$TEXMFLOCAL} 目錄。因為 \ConTeXt\ Minimals 升級包是一個 zip 壓縮檔,升級時會將該壓縮檔下載到 \type{$TEXMFLOCAL} 目錄並進行解壓縮。

\CONTEXT\ Minimals 的升級很簡單,那就是再重裝一遍。

\section{若需要 \TeX\ 與 \LaTeX}
\bookmark{若需要 TeX 與 LaTeX}

\ConTeXt\ Minimals 可以與其它 \TeX\ 發行套件友好共存。最近,我打算使用 \XeTeX\ + plain \TeX\ 格式寫一份文檔,但是 \ConTeXt\ Minimals 默認未提供 \XeTeX\ 的 plain \TeX\ 格式並且缺乏一些 plain \TEX\ 必須的字體,因此基於 \XeTeX\ 的 plain \TEX\ 及 \LATEX\ 環境均無法正常運行。對於這個問題,我能想到的比較輕省的解決方案是安裝一個 \TeX\ Live 2008,使用它所包含的 plain \TeX\ 環境,另外再安裝一個 \LATEX\ 環境以備不時之需。

首先從清華的 CTAN 鏡像 FTP 服務器下載 \TEX\ Live 2008 的網絡安裝程序並解包:

\starttyping
$ wget ftp://ftp.tsinghua.edu.cn/mirror/CTAN/systems/texlive/\
tlnet/tldev/install-tl-unx.tar.gz
$ tar -zxvf install-tl-unx.tar.gz && cd install-tl-unx
\stoptyping

\TEX\ Live 2008 的安裝程序兼顧了不同用戶的軟件使用習慣,同時提供了文本界面與圖形界面兩種安裝模式:

\starttyping
$ install-tl    # 文本界面安裝
$ install-tl --gui # 圖形界面安裝模式
\stoptyping

\type{install-tl} 程序在運行時會自動訪問預設定的 CTAN 鏡像 FTP 服務器,如果連接失敗,安裝過程也就終止了。對於國內用戶,可以通過 \type{--location} 參數將 CTAN 鏡像服務器設為清華的 FTP:

\starttyping
$ install-tl --location=ftp://ftp.tsinghua.edu.cn/\
mirror/CTAN/systems/texlive/tlnet/tldev
\stoptyping

如果網速足夠,那麼應當很快就可以在終端中顯示文本安裝界面了,具體的安裝過程應當參考 \TEX\ Live 2008 中文指南\footnote{http://tug.org/texlive/doc/texlive-zh-cn/}。\TEX\ Live 2008 成功安裝後,只要不在終端中開啟 \ConTeXt\ Minimals 的 \ConTeXt\ 環境,就不會影響它的使用。

\section[helloworld]{Hello World}

將下面代碼使用文本編輯器保存為 hello-world.tex 文件,這就是一個最簡單的 \ConTeXt\ 源文檔。

\starttyping
% hello-world.tex
\starttext
Hello World.
\stoptext
\stoptyping

我們可以在終端中使用 \type{context} 命令將這份源文檔編譯、輸出為 PDF 文檔:

\starttyping
$ . ctx
$ context hello-world
\stoptyping

上述命令將在當前目錄產生 hello-world.pdf 文檔,使用 PDF 閱讀器查看它,可以看到文檔中除了「Hello World.」字串之外,就剩下一個置於頂端的頁碼了。

在編譯 hello-world.tex 時,生成了許多中間文件,如果你不喜歡看到它們,可以使用以下命令將其清除:

\starttyping
$ context --purge
\stoptyping

\section{\TeX\ 與 \ConTeXt\ 的關係}
\bookmark{TeX 與 ConTeXt 的關係}

在上一節 hello-world.tex 文檔的編譯過程中,我們可以獲得的直觀認識如下圖所示:

\setupFLOWcharts[nx=3,ny=1,
  dx=2\bodyfontsize,dy=4\bodyfontsize,
  width=8\bodyfontsize, height=4\bodyfontsize,
  maxwidth=.9\textwidth]

\startFLOWchart[context-flow]

\startFLOWcell
\name{texfile}\location{1,1}\shape{singledocument}
\text{\ConTeXt\ 源文檔}
\connection[rl]{context}
\stopFLOWcell

\startFLOWcell
\name{context}\location{2,1}\shape{}
\text{\ConTeXt\ 程序}
\connection[rl]{pdf}
\stopFLOWcell

\startFLOWcell
\name{pdf}\location{3,1}\shape{singledocument}
\text{PDF 文檔}
\stopFLOWcell

\stopFLOWchart

\placefigure[force][fig:context-flow]{\ConTeXt\ 文檔編譯過程的直觀認識}
            {\FLOWchart[context-flow]}

實際上是沒有所謂的「\ConTeXt\ 程序」的,真正實現 \TeX\ 文檔編譯的的是 \TeX\ 引擎。在 \ConTeXt\ MkIV 中,\type{context} 程序只是一個腳本文件,只有一行代碼:

\starttyping
mtxrun --script context "$@"
\stoptyping

\type{mtxrun} 是一個 Lua 腳本,它需要調用 \LUATEX\ 引擎並讀入 cont-en.fmt 文件才可以對 \ConTeXt\ 文檔進行編譯處理。cont-en.fmt 是 \ConTeXt\ 的格式文件,可以將它理解為一本詞典,當編譯 \ConTeXt\ 文檔時,Lua\TeX\ 引擎遇到文檔中的排版命令時,就去 cont-en.fmt 文件中查找這些排版命令的定義並依命行事。可還記得 2.3 節中使用 \type{context --make} 命令麼?這個命令就是用於產生 cont-en.fmt 之類的格式文件的。這就是為什麼我們經常說 \ConTeXt\ 或 \LaTeX\ 是 \TeX\ 的一種格式的原因。

\section{讓 MkIV 講中文}

要讓 \ConTeXt\ MkIV 講中文,首先要讓它能夠找到中文字體所在。還記得 \in{}[installmkiv]\ 節中我們寫的那個 \ConTeXt\ 環境配置文件 ctx 麼?其中有一行代碼如下:

\starttyping
export OSFONTDIR="/usr/share/fonts/{adobe,winfonts}"
\stoptyping

\type{$OSFONTDIR} 可以讓 Lua\TeX\ 知道系統所安裝的字體所在。

我在 \type{/usr/share/fonts/adobe} 目錄中存放了 4 款 Adobe OTF 中文字體,在 \type{/usr/share/fonts/winfonts} 目錄中存放了一些從 Windows XP 中複製過來的一些中文字體。

有了中文字體,還需要一份字體配置文件。\ConTeXt\ 提供了一套比較高級的字體機制,叫做 Typescript,它可以設置三種字型:襯線 (Serif)、非襯線 (Sans)與等寬 (Mono) 字型。這三種字型均為英文字體機制中的術語,就像我們中文字體有楷體、宋體、黑體等類型一樣。\ConTeXt\ 的 Typescript 機制牽涉太多的字體知識,對它們的詳細描述已超出我之所能,這裡僅給出一份我照貓畫虎搞出來的的一份字體配置文件——\type{zhfonts.tex},它是隨這份文檔一起發佈的,只需將它放到 \type{TEX} 目錄樹中,譬如:

\startniceTEX
$ mkdir -p $TEXMFLLOCAL/tex/context/third
$ mv zhfonts.tex $TEXMFLLOCAL/tex/context/third
\stopniceTEX

\noindent 然後執行:

\starttyping
$ context --generate
\stoptyping

\noindent 刷新一下文檔數據庫,就可以使用該字體配置文件了。

在 zhfonts.tex 中,我使用了 AdobeSongStd-Light、AdobeHeitiStd-Regular 以及 AdobeKaitiStd-Regular 字體作為主要的中文字體。如果你沒有這些字體,可以將其替換為其它 TTF 或 OpenType 中文字體,但是要保證 \LUATEX\ 可以找到它們。由於中文字體所包含的英文字形通常比較醜陋,因此在 zhfonts.tex 文件中利用了 MkIV 的虛擬字體機制,將 lmroman10-regular、lmsans10-regular、lmmono10-regular、lmmono10-italic 以及它們的粗體的英文字形區域分別注入到相應的中文字體中。

現在,將 \in[helloworld] 節中的 hello-world.tex 文檔修改為:

\startTEX
\usetypescriptfile[zhfonts]
\usetypescript[myfont]
\setupbodyfont[myfont,rm,11pt]
\starttext
世界,你好!
\stoptext
\stopTEX

重新編譯這份文檔,即可得到中文版的 hello-world.pdf 文檔。

\tex{usetypescriptfile} 命令用於加載 typescript 文件,要求 typescript 文件與 helloworld.tex 文件均在同一目錄或者前者位於 \ConTeXt\ Minimals 的某個可以檢索到的目錄中。

\tex{usetypescript} 命令表示使用 typescript 文件中定義的 typeface。所謂 typeface,就是一個字型族。在 zhfonts.tex 文件中所定義的 typeface為 myfont,該字型族包含了 3 種字型:Serif、Sans 與 Mono。\tex{setupbodyfont} 命令是用於定義 hello-world.tex 文檔的默認字體,也就是正文字體,這裡是將 myfont 字型族中的 Serif 字型作為文檔的默認字體,並且默認尺寸為 11pt。

現在,MkIV 基本上已經解決了中文排版問題,我們應當為此感謝 Hans、Taco 等開發者的辛勤勞動。另外,還要感謝 Wang Yue、SDE 諸位同學,他們提出了許多有關 \ConTeXt\ 中文支持的要求並報告了許多 bug。特別是 Wang Yue 同學,自去年始,一直在關注 Lua\TeX\ \& \ConTeXt\ MkIV 等項目的發展,並積極地與開發者們溝通,使得開發者認真考慮了有關中文處理的諸多需求。

\section{開始 \ConTeXt\ 的旅程}

\ConTeXt\ MkIV 中文支持的成功嘗試,奠定了我開始學習 \ConTeXt\ 的信心。\CONTEXT\ 的文檔非常豐富,不過大部分都是英文的,這多少讓我有點沮喪。雖然讀英文文檔不是什麼大障礙,但是總沒有中文文檔來得親切。現狀如此,只好忍了。

目前,\ConTeXt\ 的大多數文檔是針對 MkII 版本的,除了 MkIV 手冊之外,似乎再也沒有其它文檔來講述 MkIV 的應用。不過,由於 MkIV 改變的主要是 \CONTEXT\ 底層,用戶界面基本上沒有改變,對於中文用戶而言,基本上只需將 MkII 的中文字體支持機制改換為 MkIV 的即可,這樣原來適用於 MkII 的文檔基本上也適用於 MkIV。現在,\ConTeXt\ 項目組現在正在組織進行新的 \ConTeXt\ 文檔撰寫,這個工程非常浩大,其意在於將這麼多年零散的文檔彙總到一起,並追隨 \CONTEXT\ 的最新進展,估計要到明年方能竣工。遠水解不了近渴,現在學習 \CONTEXT,還是要從 MKII 版本的文檔開始。

作為初學者,{\it\ConTeXt, an excursion}\cite[ms-cb-en]文檔是官方製作的新手入門必讀文檔。\ConTeXt\ 的官方文檔通常是分為屏幕閱讀版本與打印版本,我讀的 {\it\ConTeXt, an excursion} 是屏幕閱讀版本,第一次打開這份文檔,就驚愕了一下,沒想到 PDF 文檔居然可以做得如此精緻。內容淺顯易懂,文檔版面美觀,我覺得 \ConTeXt\ 是非常禮遇初學者的。

當讀過 {\it\ConTeXt, an excursion} 時\footnote{實際上我是用到什麼就去看什麼,現在大概看了有一半內容。},就可以使用 \ConTeXt\ 排版手頭上正在寫作的一些文檔,其實這才是真正的開始學習 \CONTEXT。對於其它任何一種 \TEX\ 都是如此,只有真正開始使用它,才有可能掌握它。一開始,文檔的版面醜點沒關係,畢竟內容是最重要的。隨著對 \CONTEXT\ 的認識日益深刻,你的文檔便會愈發具備可觀賞性。在實踐的過程中,\CONTEXT\ 用戶手冊\cite[cont-eni]是主要的參考書。當你遇到不解的命令/參數,或者想瞭解一些命令更多的細節,都可以使用 PDF 閱讀器打開 \CONTEXT\ 用戶手冊,利用閱讀器提供的查詢功能找到你所感興趣的內容,並掌握它們。

字體在排版中是決定版面美觀的重要因素,閱讀 \CONTEXT\ 新手冊中已經基本完工的 {\it fonts}\cite[fonts] 部分,便可以理解 \CONTEXT\ 的字體機制,你將獲得可以在 \CONTEXT\ 自如使用你喜歡的字體的能力;同時,我製作的那份 MkIV 中文字體配置文件也不再會令你感到困惑。

若使用 \ConTeXt\ 排版時存在向量圖繪製需求,推薦使用 \METAPOST。不過,在 \ConTeXt\ 中,我們要面對的不是 \METAPOST,而是 \METAFUN\cite[metafun],後者對前者進行了封裝,實現了 \CONTEXT\ 與 \METAPOST\ 的完美結合。{\it\ConTeXt, an excursion} 文檔版面之所以美觀,是因為在排版中大量使用 \METAFUN 來美化版面佈局的結果。

若你對 \ConTeXt\ 已經有所認識,那麼 \ConTeXt\ Wiki\footnote{\type{http://wiki.contextgarden.net/Main_Page}} 是非常好的網絡學習資源。若有能力,可以在該 Wiki 上與整個世界的 \ConTeXt\ 用戶分享你的學習經驗;若暫時沒能力,可以選擇潛水。目前,我是這個 Wiki 眾多潛水員中的一名。

目前,\CONTEXT\ MkIV 也許還存在許多的 bug,當你發現它們,應當提交 \CONTEXT\ 郵件列表\footnote{http://archive.contextgarden.net/list/context.en.html}中;當你使用 \CONTEXT\ 時有不解的問題,也可以去郵件列表中尋找答案。說來慚愧,我英文不好,並且不熟悉郵件列表的交流規則,所以一直都膽怯於郵件列表上的交流。

介紹了以上學習資源之後,我開始如釋重負。以後再在這份文檔上摺騰,心裡就不會總是想著自己在寫一本 \CONTEXT\ 教程,那樣子很累,並且很容易喪失樂趣。這份筆記的真正的意圖是自我備忘,將它投放到網絡上,是希望我的經驗能夠對他人有所幫助並且能夠一同學習、討論 \CONTEXT\ 排版知識。

\vfil
\midaligned{\color[gray]{\bfa Just for fun!}}

\stopcomponent

