\startcomponent ch1
\product ctxnotes

\chapter{對 \TEX\ 的一些認知}

本章可讀,可不讀。讀,是為了讓自己更明白 \TEX\ 是什麼,並且瞭解一下 \CONTEXT\ 的概況。不讀,是因為取這樣標題的章節通常包含了許多廢話。

「\TEX\ 是什麼」這一節,是 Hans Hangen 寫的\footnote{原文見:\type{http://www.tug.org/pracjourn/2005-3/walden-whatis/hagen-4.pdf}},我只是按照自己的理解翻譯了過來。之所以讓它在本章出現,是因為我認為作為 \CONTEXT\ 的主要開發者,Hans Hagen 對 \TEX\ 具有深入客觀的認識,他有資格告訴我們一些 \TEX\ 的真實面目,而不是那些在網絡上流傳的諸多神話。另外,在本章中,我還記述了自己對 \CONTEXT\ 的一些認識,並對這份文檔的寫作意圖進行些許交代。

\section{\TEX\ 是什麼}

在此,我認真思考了一些人提出的一些問題。這篇文章與其說它是一份評論,倒不如說它像一個開關,可以觸發更為深刻地討論。如果你只是想知道什麼是 \TeX,你可以直接讀「\TeX\ 是什麼……以及我為什麼喜歡它」那一節。

\subsubject{所有的 \TeX\ 都是平等的……\\ \rightaligned{……只是有些 \TeX\ 比其他 \TeX\ 更加平等而已}}

對於什麼是 \TeX\ 這一問題實在很難給出明確的答案。例如 Peter Flom 使用「\LaTeX\ 是……」作為開場白,將 \TeX\ 程序等價為 \LaTeX\ 這個宏包。這種等價隨處可見,因此許多用戶並不知曉 pdf\TeX(程序) 與 pdf\LaTeX(宏包)的區別。在許多系統上,如果沒有顯示的指明 pdf\TeX\ 所使用的宏包,那麼它會調用 plain \TeX\ 格式,這聽起來就更讓人混亂了。

\TeX\ 的雙重身份也常常令人混亂,它即是一種排版語言,又是一個解釋器/排版工具。\TeX Book 這本書中不僅講述了 \TeX\ 的雙重身份,還涉及了 plain \TeX\ 格式。因此,對於 \TeX\ 而言,我們有必要區分:

\definition{語言} \TeX\ 排版命令,由基本命令與宏命令構成,還有一些擴展命令,如 e\TeX\ 添加的命令\par
\definition{程序} \TeX(最早的\TeX\ 解釋器)、pdf\TeX、\XeTeX、Lua\TeX\ 等 \TEX\ 語言解釋器\par
\definition{宏包} Plain \TeX、\AMSTEX、\LaTeX、\LAMSTEX、\ConTeXt\ 等基於語言實現的宏命令集

\LaTeX\ 用戶有多種調用 \TeX\ 程序的方式:

\definition{latex} 可預裝入 \LaTeX\ 宏包的 (pdf)\TeX\ 引擎\par
\definition{pdflatex} 同上,不過它是直接輸出 pdf 文檔\par
\definition{xelatex} 由 \XeTeX\ 引擎預裝入的 \LaTeX\ 宏包\par
\definition{lambda} 由 ALEPH 或 OMEGA 引擎預裝入的 \LaTeX\ 宏包

對於 \ConTeXt\ 用戶而言,\LaTeX\ 調用 \TeX\ 程序的經驗就用不上了,因為 \ConTeXt\ 提供了 texexec 腳本,可以使用統一的方式實現不同 \TeX\ 程序的調用,譬如:

\definition{texexec --pdf somefile.tex} 使用預裝入 \ConTeXt\ 宏包的 pdf\TeX\ 引擎\par
\definition{texexec --xtx somefile.tex} 同上,不過這次調用的是 \XeTeX

很久以來,\TeX\ 引擎產生 DVI 格式輸出,然後借助一些程序可以將其處理成 PostScript 文檔(適合打印或屏幕閱讀)。由於 PDF 文檔格式日趨流行,因此有必要再使用一些工具將 PostScript 文檔轉化為 PDF 文檔。現在,使用 pdf\TeX\ 引擎可以直接由 \TeX\ 源文檔生成 PDF 文檔,所以也就沒有必要再延用過去的那種繁瑣的文檔生成方式了。

無論你怎樣使用 \TeX,都應當搞清楚你所調用的命令的相關術語的確切概念,否則當你你與他人進行 \TeX\ 知識交流或者去 BBS 尋求幫助時,別人未必使用與你相同的命令。比如在較早的 \TeX\ 發行套件中,pdf\TeX\ 默認輸出的文檔格式是 DVI,除非你顯示指定 PDF 格式輸出;而在新的 \TeX\ 發行套件中,pdf\TeX\ 默認輸出的文檔格式是 PDF。怎麼樣,現在還感到困惑麼?

\subsubject{\TeX\ 可以生成精美的文檔……\\\rightaligned{……但它不會給你什麼承諾}}

有一些 \TeX\ 的擁護者,他們為了讓更多的人接受\TeX,便經常地讚美 \TeX\ 排版的精美。這種行為過於一廂情願了。TeX排版的精美是沒有什麼疑問的,但是許多文檔看上去很 \TeX\ 化,就像 MS Word 文檔看上去很 word 化,QUARK 文檔看上去很 quark 化。在風格的變化方面,字體及其格式的影響不是很大,而預定義的模板被重複地使用,這些都導致了文檔的版面千篇一律。例如,有些 \TeX\ 用戶經常取笑使用 PowerPoint 演示文檔(因為通過這些演示文檔的一些特徵就可以辨識出來這是 PPT 文檔),卻從未意識到他們自己也是那麼幹的。許多 \TeX\ 用戶宣稱 \TeX\ 可以有效地自動處理段落文本斷行問題,但是他們往往又不得不對那些很滑稽的行間距做出寬容的姿態,這些行間距通常是自己所使用的一些命令與那些嘗試一攬子解決所有問題的宏包發生衝突時所致。由於 \TeX\ 在文本對齊方面做的非常出色,因此一旦文檔的版面發生單詞越出邊界的現象,就會非常得顯眼,在網絡上張貼的許多文檔經常向我們展示這一「特效」。

\subsubject{\TeX\ 是易於使用的……\\\rightaligned{……但需要你付出努力}}

\TeX\ 能夠很聰明地處理圖形與字體,但是郵件列表裡 (BBS) 的諸多帖子昭示著這並非是小事情,即便對於那些 \TeX\ 老手而言。\TeX\ 可以是一個易於使用的排版系統,但是用戶如果要真正的駕馭它,也必須要經歷一個痛苦的過程。有些東西純粹是排版技巧,無關乎你使用的是哪一種排版系統。

使用 \TeX\ 能得到一個極大的好處,那就是 \TeX\ 用戶能夠普遍地熱心幫助新手。由於 \TeX\ 用戶大都出於自己的意願選擇了 \TeX,因而他們通常也能夠付出足夠多的努力來掌握它。

有關 \TeX\ 及其宏包的用法,我們能夠在網絡或書店裡找到許多的學習資源(手冊、答疑、Wiki、郵件列表、新聞組等等)。

\subsubject{成也 \TeX……\\\rightaligned{……敗也 \TeX}}

在許多計算機語言中,程序員不得不明確地告訴機器有一些文本要輸出。但是 \TeX\ 卻不同,\TeX\ 對於文本的任何修飾都可能會變成可見的。有人嘲笑 MS Word 排版的文檔會出現一些不協調的空區,譬如一些重複的空格。事實上,使用 \TeX,如果你不瞭解一些宏標記的具體細節,也很容易引入許多很可笑的空區,導致頁面亂糟糟的。總之,在做一些宣判或開一些玩笑時,一定要小心謹慎才是。

在討論 MS Word 與 \TeX\ 的區別時,有人說 MS Word 從來也搞不清楚在何種情況下需要切換字體,比如一個粗體顯示的單詞之前的空格是否也要粗體顯示?但實際上 \TeX\ 在這方面並不比 MS Word 高明多少。

我們來看一下在 \TeX\ 中如何能讓一個段落變的狹窄一些,如下:

\startTEX
\def\StartNarrow{\bgroup\leftskip1em\rightskip1em\relax}
\def\StopNarrow {\egroup}
\StartNarrow
some lines of text
\StopNarrow
\stopTEX

像上面這樣的處理貌似可以將段落文本放置於一個已經作了限制的區域中。但實際上,上面的代碼並不能得到一個狹窄的段落,除非你顯式地添加上段落終止標記:

\startTEX
\StartNarrow
some lines of text\par
\StopNarrow
\stopTEX

對於這一問題,最好的解決方案是修改 \type{\StopNarrow} 宏定義:

\startTEX
\def\StartNarrow{\bgroup\leftskip1em\rightskip1em\relax}
\def\StopNarrow {\par\egroup}
\stopTEX

有許多間距都與採用這種方式得到的特徵以及並不總是很清楚的代碼效果有關。適用於這一份文檔的樣式並不見得就適合其它文檔。所有的一切都依賴於你的 \TeX\ 是如何設定的以及宏包的作者如何協調好他們的工作。

\subsubject{\TeX\ 是穩定不變的……\\\rightaligned{……汗……我們真的希望是這樣?}}

Don Knuth 一廂情願地認為 \TeX\ 程序的功能能夠適應性地擴展,去解決那些當前解決不了的問題。在優秀的老 \TeX\ 中有兩個擴展的例子:特效 (Special) 與著述 (Write)。所謂特效,就是提供一種方法去控制 \TeX\ 引擎以實現一些特殊效果,比如顏色或向圖形插入之類。如果沒有特效這一擴展,那麼 \TeX\ 用戶就要面臨很大的麻煩,需要手動去做圖片複本的剪切、黏帖之類的事情。\TeX\ 的著述擴展在撰寫文獻方面非常有用,它提供了目錄、交叉引用以及其他特徵,這些特徵的實現都需要一種反饋回路的運行機制。這兩種擴展都是以 \TeX\ 宏包的方式實現的,因此 Don Knuth 滿懷信心地認為TeX引擎不需要修改,它可以通過宏包的形式不斷地充實自身的功能。

事實上,有一些非 Knuth 式的擴展,但並不是很多。因為,鮮有人能不辭勞苦去寫一個用於化學領域文檔的子排版系統去與數學排版子系統並駕齊驅。許多擴展大都是採用 \TeX\ 宏的形式來實現的。迄今為止,沒有人對文本行號統計、並行輸出以及人性化及多語言兼容等方面提出健壯可靠的擴展方案。這又一次不得不借助 \TeX\ 宏開發的方式來實現,這是一種很髒的方式。你可能大呼慶幸,因為那些出版商沒有這些需求。

無論 \TeX\ 引擎的實現有多麼完美,總是有人希望它能夠繼續改進。目前,最值得稱道的一些 \TeX\ 擴展程序 \ETEX、\PDFTEX、\XETEX。\ETEX\ 提供了一些額外的編程功能。\PDFTEX\ 將 \TeX\ 推到了 21 世紀,提供了邊注字距調整與視覺縮放優化功能,另外還實現了完善的 PDF 輸出功能。\XETEX\ 使 \TeX\ 具備了處理 Unicode 編碼與 OpenType 字體的功能。實踐證明,只有不斷改進 \TeX\ 引擎,才可以保證 \TeX\ 不會被時代遺棄。\footnote{由於 Hans 撰寫這篇文章比較早,現在又有了一個新的 \TeX\ 引擎---\LUATEX,本文便是使用基於 \LUATEX\ 引擎的 \ConTeXt\ 編譯而得到的。\LUATEX\ 項目與 \XETEX\ 項目所要解決的基本問題是一樣的,但是前者實現了 Lua 語言在 \TeX\ 引擎的嵌入,使得用戶更靈活地擴展 \TeX\ 引擎的功能。}

當然,\TeX\ 宏包也扮演了非常重要的角色。因為無論 \TeX\ 引擎怎麼變化,但這些宏包基本上還可以照常運行。這也就是說,很久以前寫的 \TeX\ 文檔,利用擴展之後的 \TeX\ 引擎還可以正常編譯輸出成適合印刷或適合屏幕閱讀的文檔。

有人讚美 \TeX\ 系統鮮有 bug,但是對於 23 年之前使用 \type{\leaders} 排版命令的 \TeX\ 文檔,使用現在的 \TeX\ 引擎就無法再編譯了。因為在這期間,\TeX\ 引擎的一些 bug 得到了修正,因此對文檔的處理機制多少都有些變化,儘管這些變化非常之小。不過,通常而言,說 \TeX\ 系統鮮有 bug 也不為過,這種說法只是在你不刻意讓今天的 \TeX\ 系統來處理很久很久之前的文檔或使用很久以前的宏包的前提下才成立的。順便說一下,也有一些有關程序穩定不變的例子,譬如計算機語言編譯器與解釋器,實際上 \TeX\ 本身就是一種計算機語言+編譯器/解釋器。

\subsubject{\TeX\ 是什麼……\\\rightaligned{……以及我為什麼喜歡它?}}

\TeX\ 是一個允許你創建屬於你自己的排版環境的系統。在它所存在的這 20 多年來,出現了許多的排版環境,譬如 \LaTeX\ 與 \ConTeXt,它們讓用戶可以更方便的使用 \TeX。你可以根據自己的需要對它們進行功能上的擴展或者決定堅持使用它們所提供的功能,這完全取決於你個人。也有一些功能是你難以駕馭的,但這也是一個功能豐富的系統所帶給你的一個必然結果。

如果你能夠堅守著你所使用的排版環境的規範,比如保持你的文檔源碼清晰,那麼你的文檔就能夠耐得住時間的考驗。如果你採用結構化方式編輯文檔,以抽像的方式定義文檔的排版佈局,那麼最終可以實現在任何平台上都能夠得到最終的編譯輸出結果。你可以將文檔撰寫任務分配給其他人,你們一起協同工作,這種協同工作方式有助於實現郵件列表的支持,形成一個氣氛友好的社區、用戶群體,實現書籍與手冊的撰寫。如果你想完全駕馭 \TeX\ 系統(譬如 \LaTeX\ 或 \ConTeXt),這需要耗費一段時間來掌握它們。學習週期過於漫長,這似乎是個難題,但是對於很對用戶而言,他們終生都在使用 \TeX,並得益於此,所以學習週期的問題不再是問題。Don Knuth給予了我們創造精美文檔的能力,但是你需要付出一定的努力才能夠掌握它。Don Knuth 也給出了一個重要的時間界限條件,即在 100 年後,印刷技術有了極大的進步,這些使用 \TeX\ 標記所寫的文檔依然可以使用 \TeX\ 引擎進行有效處理。我們所寫的 \TeX\ 文檔,能夠生存這麼長久,這本身就是一件很讓人舒服的事情。只是要小心,在這麼漫長的時間裡,你可能會碰到一些麻煩,要避免它們,就需要保持一種開放的心態去面對 \TeX\ 的缺陷、神話以及一些奇怪的解決方案。


\section{我對 \CONTEXT\ 的認知}

\CONTEXT\ 是荷蘭 Pragma-ADE 公司基於 \TEX\ 實現的一種高端文檔製造工具,使用它可以製作非常精美的 PDF 文檔,適用於科技文檔排版與屏幕演示文檔製作。與 \LATEX\ 相比,\CONTEXT\ 的開發更為集中、活躍與激進。

\CONTEXT\ 的版本可以分為 MkI、MkII 和 MkIV。MkI 的用戶界面是荷蘭語,並且只有開發者可以看到用戶界面的具體實現。MkII 將用戶界面替換成英文,並且開放了一些用戶界面的實現,便於用戶參與開發。MkIV 是新一代 \CONTEXT,其中許多模塊重新實現了,最具革命性的是引入了 \LUATEX\ 引擎。\LUATEX\ 是 pdf\TEX\ 的一個擴展版本,其中植入了 Lua 語言,這意味著在 \TEX\ 文檔中可以使用 Lua 完成一些程序,使 \TEX\ 文檔演進成一種真正的文檔排版編程語言。另外,\LUATEX\ 提供對本地 TTF \& OTF 字體的直接支持,對於中文用戶而言,困擾大家多年的中文字體嵌入的問題算是得到很好地解決。所以,\LUATEX\ 似乎是可以結束目前 \TEX\ 引擎版本混亂、功能落後的最理想的解決方案。

事實上 \CONTEXT\ 還有一個 MkIII 版本,這是為 \XETEX\ 引擎預留的。\XETEX\ 原本是 Mac OS 平台上的一個 \TEX\ 引擎項目,不過現在 Linux、 Windows 平台都有其移植版本。\XETEX\ 所要解決的問題與 \LUATEX\ 差不多,但前者沒有像後者那樣提供一種內嵌的腳本語言。目前的 \XETEX\ 已經可以較為穩定地運行了,而 \LUATEX\ 還處於 Beta 版本,據說今年夏天會正式發佈\footnote{在寫這份文檔的時候,今年夏天快要過去了。}。現在,\XETEX\ 的最新版是 0.999,已經可以支持 UTF-8 編碼以及本地 TTF \& OTF 字體調用,用於中文文檔處理基本上沒有什麼問題了。特別對於 \LATEX\ 的中文用戶,由於 C\TEX\ 論壇上的 mytex、yindian 所提供的 XeCJK 與 zhspacing 宏包,已經可以讓使用 \XETEX\ 引擎的 \LATEX\ 得以完美地支持中文排版。

現在,\PDFTEX\ 項目已併入 \LUATEX\ 項目中,這宣告著在今後一段很長的時間裡 Knuth \TEX、\LUATEX\ 與 \XETEX\ 三足鼎立的時代的來臨,不過 Knuth \TEX\ 引擎存在的意義也許僅在於兼容以前的文檔或留給後人去考古或者兼容歷史遺留文檔。

幾乎所有的 \TEX\ 發行版中都包含了 \CONTEXT\ 模塊,但是若想更容易地使用 MkIV 版本,建議安裝 \CONTEXT\ Wiki\footnote{\type{http://wiki.contextgarden.net}} 上提供的 Minimals (\CONTEXT\ 最小發行版)。\CONTEXT\ Minimals 僅提供了運行 \CONTEXT\ 環境所需要的軟件包,所以如果你想使用 \LATEX,還請安裝其它 \TEX 發行版,譬如 \TEX\ Live,只需不安裝其中的 \CONTEXT\ 模塊即可,因為 \CONTEXT\ Minimals 與其它 \TEX\ 發行版可以友好地共存。

\section{關於這份文檔}

這份文檔僅僅是我個人學習 \CONTEXT\ 過程中所獲取的一些我認為比較重要的知識的彙總,它也許有些凌亂,不是面面俱到,有些知識講述地過於淺薄甚至出現了認識錯誤,這都是因為我還在學習中。如果您恰好是一個有經驗的 \CONTEXT\ 用戶,恰好看見了錯誤,恰好又不吝賜教,你就是我的一字之師啊,(突然提高聲音)我記你一輩子!\footnote{出自武林外傳第五十回。}

由於自知能力有限,我沒有將這份文檔寫成一部 \CONTEXT\ 中文教程的慾望,因此這份文檔並不能教會你使用 \CONTEXT。官方的文檔非常全面,如果想掌握 \CONTEXT,我推薦你去閱讀它們。其實在這份文檔中,每當我碰到一些細節知識沒有耐心去講述時,便會偷懶,往往會裝作很耐心地告訴你在 \CONTEXT\ 手冊的哪一節可以找到詳細的記述。

我寫這份文檔的出發點與你之所以讀到它是一樣的,都是出於對 \CONTEXT\ 的喜愛,並且希望它可以化作自己的如摶之筆,用以書寫優美的文檔。當我開始寫這份文檔的時候,還是一個菜鳥。待得這份文檔越來越完整,越來越豐滿的時候,也許我就變成了一個有經驗的菜鳥。我這麼說,不是謙虛,而是面對著 \TEX\ 的博大,看見了自己的渺小。

本文檔有許多內容是只適於 Linux 環境,特別是關於 \CONTEXT\ 編譯環境搭建方面的內容。Windows 或 Mac OS 環境中的 \CONTEXT\ 我從未嘗試過,並且也不想去嘗試。在我看來,\TEX\ 在 Linux 終端裡運行,再配合一些文本處理工具,儼然如魚得水一般,所以我一點都不嚮往 GUI 的陸地。

\stopcomponent

